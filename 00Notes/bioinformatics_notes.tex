\documentclass[12pt, lettersize]{article}

\usepackage[utf8]{inputenc} %uft8 is usually the std encoding

\usepackage{graphicx} %to insert images
\graphicspath{ {}{} }%specify image location path. 
%leave empty if images are on the same folder as the .tex source.

\usepackage{comment}
\usepackage{color}

\usepackage{enumerate}
%to use personalized enumerations 
%usage: \begin{enumerate}[*listing style you want. ex: i*]
%content...
%\end{enumerate}

\usepackage{geometry}

\geometry{
	total={170mm,230mm},
	left=20mm,
	top=20mm,
}

%Glossary definition and hyperref 
\usepackage{hyperref}
\usepackage{glossaries}
\makeglossaries

\newglossaryentry{heterozygous}
{
	name={heterozygous},
	description={description}
}


%title page definition
\setlength\parindent{0pt}

\title{Bioinformatics Notes and Terminology}
\author{Lev Cesar Guzman Aparicio \\ \texttt{lguzm038@uottawa.ca}}

%actual document begins
\begin{document}
\maketitle

%first section will concentrate on molecular biology
\section{Molecular Biology}

\subsection{DNA}

\subsection{RNA}

\subsubsection{mRNA}

%second section will focus on heredity and probability
\section{Genetics/Heredity}

\subsection{Mendel's First Law of Inheritance}

Mendel's maxim that \textsl{every} gene has two alleles, derived from each parent. 

\subsubsection{Punnet Squares}

A \textbf{Punnet Square} is a graphical way to deduce all possible combinations of alleles that may be inherited to offspring. In addition, it also tells us what percentage of the offspring contain a particular allele.



\subsection{Mendel's Second Law of Inheritance}

While Mendel's first law described the behavior among alleles that affect a single phenotype; Mendel's second law describes the relationships among alleles that affect different characteristics. \\

\textsl{\textbf{Factors (gene) for different characteristics are inherited with no dependence on each other.}} \\

The previous statement came to be known as \textbf{Mendel's Second Law} or \textbf{Law of Independent Assortment}


%some extra notes on algorithms
\section{Data Structures and Algorithms Used}

\subsection{Suffix Tree/Array}

\subsection{Skew Algorithm}

%the glossary
\printglossaries


\end{document}